\documentclass[12pt,a4paper]{article}  % Khai báo lớp văn bản
\usepackage[utf8]{vietnam} % Gói lệnh phông tiếng Việt
\usepackage{bookman}
\usepackage{amsmath,amsthm,amssymb}% Các gói lệnh về môi trường và kí hiệu
\usepackage[top=2cm, bottom=2cm, left=2.5cm, right=2cm]{geometry}% Căn lề
\usepackage{graphicx}% gói lệnh chèn hình ảnh
\usepackage[colorlinks]{hyperref} %tô màu
\usepackage{caption}%gói lệnh chèn chú thích hình ảnh
\usepackage{csquotes}% chèn trích dẫn câu nói
	\DeclareQuoteStyle[american]{english}
	{\itshape\textquotedblleft}
	[\textquotedblleft]
	{\textquotedblright}
	[0.05em]
	{\textquoteleft}
	{\textquoteright}
\begin{document} 
 \section{Tiểu sử Bill Gates} 
 \subsection{Giới thiệu sơ}
 \begin{center}
 \includegraphics[width=0.75\textwidth]{h1.jpg}
 \end{center}
  William Henry "Bill" Gates III (sinh ngày 28 tháng 10 năm 1955) là một doanh nhân người Mỹ, nhà từ thiện, tác giả và chủ tịch tập đoàn Microsoft, hãng phần mềm khổng lồ mà ông cùng với Paul Allen đã sáng lập ra. Ông luôn có mặt trong danh sách những người giàu nhất trên thế giới và gần đây, ông là người giàu thứ hai thế giới với tài sản 105,3 tỷ đô la Mỹ. Ở Microsoft, Gates làm CEO và kiến trúc sư trưởng phần mềm định hướng cho sự phát triển của tập đoàn. Hiện tại, ông là cổ đông với tư cách cá nhân lớn nhất trong tập đoàn. Ngày làm việc toàn phần cuối cùng dành cho Microsoft của Gates là ngày 27 tháng 6 năm 2008. Ông vẫn còn giữ cương vị chủ tịch Microsoft nhưng không điều hành hoạt động tập đoàn.
  \subsection{Tiểu sử}
   \subsubsection{Thời niên thiếu} 
    \begin{center}
   		\includegraphics[width=0.75\textwidth]{h2.jpg}
   	\end{center}
   Bill Gates sinh ra ở Seattle, Washington, bố là William H. Gates, Sr. và mẹ Maxwell Gates. Ông sinh ra và lớn lên trong gia đình khá giả; bố ông là một luật sư có tiếng, mẹ ông thuộc ban giám đốc của công ty tài chính. Năm 13 tuổi, ông vào học trường Lakeside, một trường dự bị cho các học sinh giỏi. Khi Gates học lớp 8, ông thấy thích thú khi lập trình trên máy tính Model 33 ASR của General Electric (GE) bằng ngôn ngữ BASIC. Ông đã viết chương trình đầu tiên trên máy tính này đó là các thao tác của trò chơi tic-tac-toe cho phép người chơi thi đấu với máy tính.
   \begin{figure}[h!]
   	\centering
   	\includegraphics[width=0.75\textwidth]{h3.jpg}
   	\captionsetup{labelformat=empty}
   	\caption{Chương trình Tic Tac Toe của Bill Gates}
   \end{figure}
	Hãng Computer Center Corporation (CCC) sáng chế ra PDP-10 đã cấm bốn học sinh trường Lakeside bao gồm Bill Gates và Paul Allen trong mùa hè sau khi bốn người này đã khai thác các lỗi trong hệ điều hành để nhận được thêm thời gian sử dụng hệ máy tính này. Khi hết hạn cấm, bốn học sinh đề nghị với công ty CCC là họ sẽ tìm các lỗi trong phần mềm của CCC và đổi lại họ được sử dụng các máy tính của công ty này. Việc thỏa thuận với CCC bị kết thúc vào năm 1970, khi công ty này bị phá sản.
	
	\noindent Một năm sau, Information Sciences Inc. đã thuê bốn học sinh trường Lakeside để viết một chương trình trả lương bằng ngôn ngữ COBOL, cho phép họ có thời gian sử dụng máy tính và bản quyền phần mềm của công ty. Sau này ông nhận xét : \enquote{It was hard to tear myself away from a machine at which I could so unambiguously demonstrate success}. Lúc 17 tuổi, Gates lập kế hoạch kinh doanh với Allen đó là Traf-O-Data nhằm đếm lưu lượng giao thông dựa trên bộ xử lý Intel 8008.
	
	\noindent Gates tốt nghiệp trường Lakeside vào năm 1973. Ông đạt được 1590 trên 1600 điểm ở kỳ thi SAT và ghi danh vào Đại học Harvard trong mùa thu năm 1973. Khi học ở Harvard, ông đã quen Steve Ballmer, người sau này kế vị chức CEO của Microsoft.
	\begin{figure}[h!]
	\centering
	\includegraphics[width=0.75\textwidth]{h4.jpg}
	\captionsetup{labelformat=empty}
	\caption{Bill Gates và Steve Ballmer}
	\end{figure}
	
	\noindent Ông liên lạc với Paul Allen, cùng tham gia vào Honeywell trong mùa hè năm 1974. Vào năm sau, chứng kiến sự ra đời của máy MITS Altair 8800 trên nền vi xử lý Intel 8080, lúc này Gates và Allen nhận ra đây là cơ hội cho họ sáng lập ra một công ty về phần mềm máy tính.
	\subsubsection{Sự sáng lập Micro-soft và PC-DOS}
	Sau khi đọc bài báo trên tạp chí Popular Electronics số tháng 1 năm 1975 về khả năng của máy Altair 8800, Gates đã liên lạc với công ty đã sáng chế ra chiếc mày này là Micro Instrumentation and Telemetry Systems (MITS), để công ty này có thể mời ông và những người khác làm việc với chiếc máy trên nền tảng trình thông dịch BASIC. Thực tế, Gates và Allen đã không được tiếp xúc với máy Altair và họ đã không viết mã chương trình cho nó; hai người chỉ muốn thử xem công ty MITS có muốn họ cộng tác hay không.
	
	\noindent
	 Giám đốc MITS là Ed Roberts đồng ý cho họ sử dụng phiên bản demo, và chỉ trong vài tuần họ đã phát triển chương trình giả lập máy Altair chạy trên một máy tính mini, và sau đó là ngôn ngữ BASIC. Cuộc thử nghiệm chiếc máy ở Albuquerque đã thành công và kết quả là một bản thỏa thuận với MITS để công ty phân phối trình thông dịch dưới tên gọi Altair BASIC. Paul Allen được mời làm việc cho MITS, và Gates đã rời trường Harvard để tới làm việc cùng Allen ở MITS tại Albuquerque vào tháng 11 năm 1975. Họ đặt tên cho sự hợp tác giữa hai người là "Micro-Soft" (Micro computer and Software) và trụ sở đầu tiên của công ty đặt ở Albuquerque. Vào ngày 26 tháng 11 năm 1976, thương hiệu "Microsoft" đã được đăng ký tại Văn phòng bang New Mexico. Không lâu sau đó, Microsoft tách ra khỏi MITS vào cuối 1976, công ty tiếp tục phát triển các ngôn ngữ lập trình cho các hệ máy khác nhau.
	
	\noindent
	Năm 1980, IBM thông qua Microsoft để viết trình thông dịch BASIC cho máy tính cá nhân sắp được tung ra của họ là máy IBM PC. Vài tuần sau, Gates đề xuất sử dụng 86-DOS, một hệ điều hành tương tự như CP/M do Tim Paterson của hãng Seattle Computer Products (SCP) viết cho các phần cứng có tính năng tương tự như PC. Microsoft đã thực hiện một thỏa thuận giá \(\$ \)50.000 với SCP để trở thành đại lý cấp phép độc quyền, và sau đó là chủ sở hữu hoàn toàn đối với 86-DOS.
	
	\noindent
	Sau khi nâng cấp hệ điều hành cho phù hợp với PC, Microsoft chuyển giao nó cho IBM với tên gọi PC-DOS với phí một lần trao đổi là \(\$ \)50.000. Và đúng là như vậy, nhờ thị phần lớn của MS-DOS làm cho Microsoft trở thành hãng phần mềm lớn trong công nghiệp phần mềm. Giữ lại source code, Gates yêu cầu các nhà sản xuất phần cứng PC, cả IBM, phải trả phí bản quyền để phần mềm để chạy MS-DOS nhờ thị phần lớn của MS-DOS làm cho Microsoft trở thành hãng phần mềm lớn trong công nghiệp, doanh thu tăng từ 7 triệu đôla năm 1980 lên đến 16 triệu đôla năm 1981. Ông giám sát quá trình tái cơ cấu Microsoft vào ngày 25 tháng 6 năm 1981, trong đó kết hợp lại công ty ở tiểu bang Washington và Gates trở thành Chủ tịch và Chủ tịch hội đồng quản trị của Microsoft. 
	\begin{figure}[h!]
		\centering
		\includegraphics[width=0.75\textwidth]{h5.jpg}
		\captionsetup{labelformat=empty}
		\caption{IBM PC-DOS User Interface}
	\end{figure}
	
	\subsection{Windows}
	\begin{center}
		\includegraphics[width=0.75\textwidth]{h6.jpg}
	\end{center}
	Microsoft phát hành phiên bản bán lẻ đầu tiên của Microsoft Windows vào ngày 20 tháng 11 năm 1985, và trong tháng 8, công ty ký hợp đồng với IBM nhằm phát triển một hệ điều hành riêng biệt gọi là OS/2. Mặc dù hai công ty đã phát triển thành công phiên bản đầu tiên của hệ điều hành mới, nhưng sự gắn kết giữa những ý tưởng sáng tạo khác nhau đã dần làm suy yếu quan hệ đối tác. Gates phân phát một bản ghi nhớ nội bộ ngày 16 tháng 5 năm 1991 tuyên bố chấm dứt sự hợp tác trong OS/2 và Microsoft sẽ chuyển sang nỗ lực phát triển nhân hệ điều hành Windows NT.
	\subsection{Phong cách quản lý}
	\begin{center}
		\includegraphics[width=0.75\textwidth]{h7.jpg}
	\end{center}
	Từ khi thành lập Microsoft năm 1975 cho đến 2006, Gates có trách nhiệm chính trong chiến lược sản phẩm của công ty. Ông đã tích cực mở rộng phạm vi sản phẩm của công ty, và ở bất cứ nơi nào Microsoft đạt được vị trí thống trị của nó thì ông mạnh mẽ bảo vệ vị thế này. Ông đạt được danh tiếng vượt xa so với những người khác; vào đầu năm 1981 một giám đốc điều hành công nghiệp phàn nàn trước công chúng rằng: "Gates có tính xấu là không chịu nghe và trả lời bằng điện thoại."
	
	\noindent Vai trò của Gates tại Microsoft trong phần lớn lịch sử của tập đoàn chủ yếu là người quản lý và điều hành. Tuy nhiên, ông cũng tham gia vào phát triển phần mềm trong những ngày đầu của công ty, đặc biệt về các sản phẩm ngôn ngữ lập trình. Ông quyết định chia trọng trách đảm nhiệm của mình, đề bạt Ray Ozzie là người quản lý hàng ngày và Craig Mundie giữ vai trò quản lý chiến lược sản phẩm dài hạn.
	\subsection{Cuộc sống cá nhân}
	\begin{figure}[!h]
	\centering
	\begin{minipage}[t]{4cm}
		\centering
		\includegraphics[width=7cm,height=4.5cm,keepaspectratio]{h8.jpg}
	\end{minipage}
	\hspace{3cm}
	\begin{minipage}[t]{4cm}
		\centering
		\includegraphics[width=7cm,height=4.5cm,keepaspectratio]{h9.jpg}
	\end{minipage}
	\end{figure}
	
	\noindent Gates cưới Melinda French ngày 1 tháng 1 năm 1994. Họ có ba con: Jennifer Katharine, Rory John, Phoebe Adele. Gia đình họ sống trong khu biệt thự nhìn ra hồ Washington ở Medina, bang Washington. Theo thống kê công khai của quận King, cho đến 2006 giá trị của khu biệt thự vào khoảng 125 triệu \(\$ \), và thuế bất động sản hàng năm là \(\$ \)991.000.
	Ông cũng đầu tư kinh doanh tại các công ty khác ngoài Microsoft, mà trong năm 2006 ông kiếm được số tiền 966.667\(\$ \) với mức lương 616.667\(\$ \) và khoản thưởng 350.000\(\$ \) từ các công ty này. Ông thành lập hãng Corbis, một công ty ảnh kỹ thuật số vào năm 1989. Năm 2004 ông là giám đốc ban quản trị của Berkshire Hathaway, một tập đoàn đầu tư do người bạn lâu năm Warren Buffett thành lập. Tháng 3 năm 2010, Bill Gates được xếp hạng là người giàu thứ hai thế giới sau Carlos Slim.
	\subsection{Công tác từ thiện}
	\begin{center}
		\includegraphics[width=0.75\textwidth]{h10.jpg}
	\end{center}
	Gates bắt đầu đánh giá cao sự mong đợi từ những người khác khi dư luận cho rằng ông có thể dùng tài sản của mình để làm từ thiện. Gates đã học cách làm của Andrew Carnegie và John D. Rockefeller, và vào năm 1994 ông bán một số cổ phiếu của Microsoft nhằm tạo dựng Quỹ William H. Gates. Năm 2000, Gates và vợ đã sáp nhập ba quỹ của gia đình thành một là Quỹ Bill \& Melinda Gates, quỹ từ thiện hoạt động công khai lớn nhất thế giới hiện nay. Cách hoạt động của quỹ cho phép các nhà hảo tâm biết được thông tin mà tiền họ quyên góp sẽ được sử dụng như thế nào, không giống như cách hoạt động của những tổ chức từ thiện lớn khác như Wellcome Trust.
	
	\noindent Quỹ đầu tư vào các công ty có mục đích làm giảm tỷ lệ đói nghèo ở các nước kém phát triển, vào các công ty sản xuất gây ô nhiễm nặng, công ty dược mà nhiều loại thuốc không được bán ở các nước đang phát triển. Mục tiêu của Quỹ là thúc đẩy những ý tưởng sáng tạo, phát triển các công nghệ năng lượng sạch, nâng cao khả năng chăm sóc sức khỏe của xã hội cũng như đầu tư vào giáo dục
\section{Đóng góp của ông trong ngành CNTT}
\subsection{MS DOS và Windows}
MS-DOS (viết tắt của Microsoft Disk Operating System, Hệ điều hành đĩa từ Microsoft) là hệ điều hành của hãng phần mềm Microsoft. Đây là một hệ điều hành có giao diện dòng lệnh (command-line interface) được thiết kế cho các máy tính họ PC (Personal Computer). MS-DOS đã từng rất phổ biến trong suốt thập niên 1980, và đầu thập niên 1990, cho đến khi Windows 95 ra đời. MS-DOS là hệ điều hành đơn nhiệm. Tại mỗi thời điểm chỉ thực hiện một thao tác duy nhất. Nói một cách khác, MS-DOS chỉ cho phép chạy một ứng dụng duy nhất tại mỗi thời điểm.

\noindent Năm 1981, tập đoàn IBM giới thiệu Máy tính cá nhân của IBM (IBM Personal Computer - IBM PC). Máy tính IBM nhanh hơn đáng kể so với các máy đối thủ, có dung lượng bộ nhớ gấp khoảng 10 lần và được hỗ trợ bởi tổ chức bán hàng lớn của IBM. Máy tính IBM đã trở thành máy tính cá nhân phổ biến nhất thế giới và cả bộ vi xử lý của nó, Intel 8088 và hệ điều hành được điều chỉnh từ hệ thống MS-DOS của Microsoft Corporation, đã trở thành tiêu chuẩn công nghiệp. Sau khi máy phát hành ra năm 1981, IBM đã nhanh chóng thiết lập tiêu chuẩn kỹ thuật cho ngành công nghiệp PC và MS-DOS cũng đẩy mạnh các hệ điều hành cạnh tranh.Các nhà sản xuất máy tính tương thích IBM, hoặc nhân bản, cũng đã chuyển sang Microsoft cho phần mềm cơ bản của họ. Đến đầu những năm 1990, ông đã trở thành nhà sản xuất vua tối cao của ngành công nghiệp PC.
\begin{figure}[h!]
	\includegraphics[width=0.75\textwidth]{h11.jpg}
	\captionsetup{labelformat=empty}
	\caption{\textcolor{blue}{Máy tính cá nhân IBM được giới thiệu năm 1981. Microsoft đã cung cấp hệ điều hành MS-DOS}}
\end{figure}
Cùng lúc đó, Apple do Steve Jobs lãnh đạo kêu gọi hợp tác với Microsoft để viết các phần mềm cho máy tính của họ. Thông qua các lần hợp tác với Apple, Bill Gates nhận ra rằng Graphical User Interface (GUI) của Macintosh sẽ tạo thiện cảm và dễ sử dụng cho người sử dụng hơn text-keyboard UI của MS-DOS. Ngoài ra, Bill Gates đã đến một buổi trình diễn tại COMDEX năm 1982 của Visi On, một giao diện người sử dụng bộ phần mềm đồ họa cho máy tính IBM.

\noindent GUI là định dạng hiển thị cho phép người dùng chọn lệnh, gọi tập tin, khởi động chương trình và thực hiện các tác vụ thông thường khác bằng cách sử dụng một thiết bị được gọi là chuột để trỏ đến biểu tượng hình ảnh hoặc danh sách các lựa chọn menu trên màn hình. Kiểu định dạng này có một số lợi thế nhất định so với các giao diện khác - trong đó người dùng gõ các lệnh dựa trên văn bản hoặc ký tự trên bàn phím để thực hiện các tác vụ thông thường. Một cửa sổ GUI, các menu kéo xuống, hộp thoại và các cơ chế điều khiển khác có thể được sử dụng trong các chương trình và ứng dụng mới theo cách được tiêu chuẩn hóa, để các tác vụ thông thường luôn được thực hiện theo cách tương tự. 

\noindent Vào ngày 10 tháng 11 năm 1983, Microsoft đã công bố Windows. Với giá \(\$ \)99, nó đi kèm với một notepad, lịch, đồng hồ, cardfile, ứng dụng đầu cuối, trình quản lý tệp, trò chơi Reversi, Windows Write và Windows Paint. Các tài liệu báo chí ban đầu, được chuẩn bị bằng Windows Write, đã có trích dẫn này từ Bill Gates: \enquote{Windows provides unprecedented power to users today and a foundation for hardware and software advancements of the next few years. It is unique software designed for the serious PC user, who places high value on the productivity that a personal computer can bring.}

\begin{figure}[!h]
	\centering
	\includegraphics[width=0.75\textwidth]{h12.jpg}
	\captionsetup{labelformat=empty}
	\caption{Giao diện Windows 1.0}
\end{figure}
\noindent Như được trích trong số tháng 12 năm 1983 của Tạp chí BYTE, Windows là một nỗ lực để làm cho hệ điều hành máy tính để bàn tương đối phải chăng. Khi hầu hết các máy tính vẫn chủ yếu dựa trên văn bản, các yêu cầu phần cứng cho hệ điều hành máy tính để bàn rất tốn kém: Apple Lisa có giá khởi điểm gần 10.000 USD và hệ thống Visi On cạnh tranh đòi hỏi một đĩa cứng đắt tiền với dung lượng trống 2,2 MB, như RAM 512KB. Thay vào đó, Windows hứa hẹn các tính năng tương tự với một cặp ổ đĩa mềm hai mặt rẻ hơn và một nửa bộ nhớ.
\begin{center}
	\includegraphics[width=0.75\textwidth]{h13.jpg}
\end{center}
Windows 1.0 được phát hành trong những đánh giá trái chiều. Hầu hết các nhà phê bình coi nó là nền tảng để có tiềm năng trong tương lai, nhưng Windows 1.0 đã không được đáp ứng mong đợi đó. Nhiều ý kiến chỉ trích yêu cầu hệ thống đòi hỏi của nó, đặc biệt chú ý hiệu suất kém kinh nghiệm khi chạy nhiều ứng dụng cùng một lúc, và rằng Windows khuyến khích việc sử dụng một con chuột để điều hướng, một khái niệm tương đối mới vào thời điểm đó. Nhìn lại, Windows 1.0 được coi là một thất bại với các ấn phẩm công nghệ hiện đại, nhưng vẫn thừa nhận tầm quan trọng tổng thể của nó với lịch sử của dòng của Windows. Nathaniel Borenstein (người tiếp tục phát triển các tiêu chuẩn MIME) và đội ngũ IT của mình tại Đại học Carnegie Mellon cũng nêu lên tầm quan trọng của Windows khi nó lần đầu tiên được trình bày cho họ bởi một nhóm các đại diện của Microsoft. Đánh giá thấp tác động trong tương lai của nền tảng này, ông tin rằng so với một quản lý cửa sổ trong nhà, "những kẻ đã đến với hệ thống thảm hại và ngây thơ này. Chúng tôi chỉ biết họ sẽ không bao giờ đạt được bất cứ điều gì"
\subsection{Microsoft}
\begin{center}
	\includegraphics[width=0.75\textwidth]{h14.jpg}
\end{center}
Có thể nói, đóng góp lớn nhất của Bill Gates đối với ngành công nghiệp chính là tập đoàn Microsoft.

\noindent Microsoft là một tập đoàn đa quốc gia của Hoa Kỳ đặt trụ sở chính tại Redmond, Washington; chuyên phát triển, sản xuất, kinh doanh bản quyền phần mềm và hỗ trợ trên diện rộng các sản phẩm và dịch vụ liên quan đến máy tính. Nếu tính theo doanh thu thì Microsoft là hãng sản xuất phần mềm lớn nhất thế giới. Nó cũng được gọi là "một trong những công ty có giá trị nhất trên thế giới". Các sản phẩm phần mềm nổi tiếng nhất là dòng hệ điều hành Microsoft Windows, bộ Microsoft Office, trình duyệt web Internet Explorer và Edge. Các sản phẩm phần cứng hàng đầu của nó là máy chơi game video Xbox và dòng máy tính cá nhân màn hình cảm ứng Microsoft Surface.

\noindent Được thành lập để phát triển phần mềm trình thông dịch BASIC cho máy Altair 8800, Microsoft vươn lên thống trị thị trường hệ điều hành cho máy tính gia đình với MS-DOS giữa những năm 1980. Kể từ thập niên 1990, công ty đã đa dạng hóa sản phẩm hệ điều hành và tiến hành nhiều thương vụ thâu tóm công ty mà điển hình là sáp nhập LinkedIn với giá 26,2 tỉ đô la vào tháng 12 năm 2016, và Skype Technologies với 8,5 tỉ đô la vào tháng 5 năm 2011. Công ty cũng cung cấp nhiều phần mềm máy tính và máy chủ cho người dùng cá nhân và doanh nghiệp, trong đó có công cụ tìm kiếm Internet (với Bing), thị trường dịch vụ số (với MSN), thực tế hỗn hợp (HoloLens), điện toán đám mây (Azure) và môi trường phát triển phần mềm (Visual Studio).
\subsubsection{Microsoft Office}
\begin{center}
	\includegraphics[width=0.75\textwidth]{h15.jpg}
\end{center}
Microsoft Office (hoặc đơn giản là Office) là tên của một bộ ứng dụng văn phòng gồm các chương trình, máy chủ, và dịch vụ phát triển bởi Microsoft, được giới thiệu lần đầu bởi Bill Gates ngày 1 tháng 8 năm 1988, tại COMDEX ở Las Vegas. Ban đầu đây là một cụm từ marketing cho một gói ứng dụng, phiên bản đầu tiên của Office gồm Microsoft Word, Microsoft Excel, và Microsoft PowerPoint.
\begin{itemize}
	\item One entry in the list
	\item Another entry in the list
\item Microsoft Word: một chương trình soạn thảo văn bản trong Microsoft Office. Phiên bản đầu tiên của Word, phát hành vào mùa thu 1983, dành cho hệ điều hành MS-DOS và đặc biệt đã giới thiệu chuột máy tính rộng rãi.
\item Microsoft Excel: một chương trình xử lý bảng tính được thiết kế ba đầu để cạnh tranh với Lotus 1-2-3, và cuối cùng thay thế nó. Microsoft phát hành phiên bản đầu tiên của Excel cho Mac OS năm 1985 và phiên bản cho Windows đầu tiên (số 2.05 để theo với phiên bản Mac) tháng 11 năm 1987.
\item Microsoft PowerPoint: một phần mềm trính chiếu dùng để tạo những bài trình chiếu gồm chữ, đồ họa và những đối tượng khác có thể được hiển thị trên màn hình để trình chiếu bằng máy chiếu.
\item Microsoft Access: một phần mềm quản lý cơ sở dữ liệu cho Windows kết hợp Hệ quản trị cơ sở dữ liệu quan hệ Microsoft Jet Database Engine với giao điện đồ họa người dùng và những công cụ phát triển phần mềm.
\item Microsoft OneNote: một chương trình ghi chú thu thập chữ viết tay hay đánh máy, hình vẽ, hình chụp màn hình và bình luận âm thanh.
\item Microsoft Outlook: một chương trình quản lý thông tin cá nhân thay thế Windows Messaging, Microsoft Mail, và Schedule+ từ Office 97, bao gồm một trình duyệt mail, lịch, quản lý công việc và địa chỉ.
\item Windows, một ứng dụng di động cho Windows Phone, iOS, Android, và Symbian, và một ứng dụng Metro cho Windows 8 trở lên.
\item Microsoft Publisher: một ứng dụng chế bản điện tử cho Windows dùng để thiết kế tờ rơi, nhãn, lịch, thiệp chúc mừng, danh thiếp, bản tin, trang web, bưu thiếp,...
\item Skype for Business: một chương trình giao tiếp tích hợp cho hội nghị và cuộc họp trong thời gian thực, nó là ứng dụng Microsoft Office desktop duy nhất phải dùng với một cơ hở hạ tầng mạng máy tính và không có từ "Microsoft" ở trước tên.
\item Microsoft Project: một ứng dụng quản lý dự án cho Windows để theo dõi tiến trình và tạo sơ đồ mạng và sơ đồ ngang Gantt, không có trong bất kì bộ Office nào.
\item Microsoft Visio: một ứng dụng sơ đồ và lưu đồ cho Windows không có trong bất kì bộ Office nào.
\end{itemize}
\subsubsection{Xbox}
Microsoft Xbox là gaming console thế hệ thứ sáu được tung ra lần đầu tiên vào ngày 15 tháng 11 năm 2001 ở Mỹ, ngày 22 tháng 2 năm 2002 ở Nhật Bản và ngày 14 tháng 3 năm 2002 ở châu Âu. Không ai nghĩ rằng Xbox sẽ thành công Gamecube của Nintendo, Dreamcast của SEGA và sự thống lĩnh thị trường của PS2 của Sony.
\begin{figure}[!h]
	\centering
	\includegraphics[width=0.75\textwidth]{h16.jpg}
	\captionsetup{labelformat=empty}
	\caption{Xbox thế hệ 1}
\end{figure}

\noindent Nhưng hóa ra là thành công ngoài mong đợi dữ dội nhất của Microsoft. Microsoft đã thua lỗ (hoặc, một cách lịch sự hơn, đã đầu tư) hơn 4 tỷ đô la trong giao diện điều khiển đầu tiên. Nhưng Xbox đã đem lại cho Microsoft một vị thế trong ngành công nghệ giải trí và hiện nay, Microsoft là 1 trong 3 công ty chiếm lĩnh thị trường game console.
\subsubsection{Microsoft Azure}
\begin{center}
	\includegraphics[width=0.75\textwidth]{h17.jpg}
\end{center}
Microsoft Azure (trước đây là Windows Azure) là một dịch vụ điện toán đám mây do Microsoft tạo ra để xây dựng, thử nghiệm, triển khai và quản lý các ứng dụng và dịch vụ thông qua các trung tâm dữ liệu do Microsoft quản lý. Nó cung cấp phần mềm dưới dạng dịch vụ (SaaS), nền tảng là dịch vụ (PaaS) và cơ sở hạ tầng dưới dạng dịch vụ (IaaS) và hỗ trợ nhiều ngôn ngữ lập trình, công cụ và khung khác nhau, bao gồm cả phần mềm và hệ thống của bên thứ ba và dành riêng cho Microsoft.

\noindent Azure được công bố vào tháng 10 năm 2008, bắt đầu với tên mã là "Project Red Dog", và được phát hành vào ngày 1 tháng 2 năm 2010, dưới dạng "Windows Azure" trước khi được đổi tên thành "Microsoft Azure" vào ngày 25 tháng 3 năm 2014. 

\noindent Microsoft Azure được ứng dụng trong các lĩnh vực như: AI + Machine Learning, Analytics, Blockchain, Compute, Containers, Databases, Developer Tools, DevOps.

\end{document}